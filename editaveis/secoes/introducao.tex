\chapter{Introdução}

Esse documenta relata a construção de um protótipo para um aplicativo chamado MyPush passando por fases de um ciclo de vida
de design de interação, com o objetivo de relatar a execução de um processo de construção de um design.

\section{Organização do Documento}

Esse documento está organizado nos seguintes capítulos: Ciclos de vida, Planejamento, \textit{Storyboard}, Requisitos,
Metas, Questionários, Protótipos, Avaliações, Ferramentas e Considerações Finais.

No capítulo \textit{Ciclos de vida} são apresentados alguns ciclos de vida que foram estudados e o ciclo de vida escolhido.

No capítulo \textit{Planejamento} é apresentado o cronograma do projeto e o planejamento das avaliações.

No capítulo \textit{Storyboard} é apresentada a \textit{Storyboard} que representa a ideia do aplicativo em alto nível.

No capítulo \textit{Requisitos} são apresentados os requisitos funcionais e não funcionais que o aplicativo deve atender.

No capítulo \textit{Metas} são apresentadas as metas de usabilidade a serem alcançadas na construção do protótipo.

No capítulo \textit{Questionários} são apresentados os questionários estudados para serem utilizados nas avaliações e os questionários
escolhidos.

No capítulo \textit{Protótipos} é descrito o processo de construção dos protótipos de baixa e alta fidelidade utilizados e são apresentadas
as versões mais recentes de ambos.

No capítulo \textit{Avaliações} são apresentadas todas as avaliações feitas durante o projeto.

No capítulo \textit{Ferramentas} estão descritas as ferramentas utilizadas.

No capítulo \textit{Considerações Finais} é apresentada a conclusão deste documento.

