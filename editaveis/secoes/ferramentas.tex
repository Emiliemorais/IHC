\chapter[Ferramentas]{Ferramentas Utilizadas}

\begin{itemize}
  \item Storyboard: \textit{BitStrips}
  
  \subitem É uma ferramenta que permite criar cenários com a interação de personagens e objetos. Dessa forma, pode-se ilustrar o ambiente e os passos envolvidos na interação entre o usuário e a aplicação.
  
  \item Prototipação: \textit{Balsamiq Mockups}
  
  \subitem É uma ferramenta online que possibilita ao usuário criar, elaborar links entre páginas e disponibilizar um esboço de seu site ou aplicação.
  
  \item Avaliação de Acessibilidade: \textit{ASES - Avaliador e Simulador de Acessibilidade de sítios}
  
  \subitem De acordo com o Portal de Governo Eletrônico	do Brasil “é uma ferramenta que permite avaliar, simular e corrigir a acessibilidade de páginas, sítios e 
  portais, sendo de grande valia para os desenvolvedores e publicadores de conteúdo.”. \cite{brasil} 
  
  \item Ferramenta de contraste: \textit{Colour Contrast Analyser}
  
  \subitem De acordo com \citeonline{cca}, essa ferramenta ajuda a determinar a legibilidade do texto e o 
  contraste de elementos visuais, tais como controles gráficos e indicadores visuais.
\end{itemize}

