\chapter{Planejamento}

Nesse capítulo é apresentado o cronograma do projeto e o planejamento das avaliações do protótipo.
\graphicspath{{figuras/}}

\section{Cronograma}
\begin{figure}[!htb]
 \centering
 \includegraphics[width = 17.5cm, height = 15cm]{cronograma.png}
 \caption{Cronograma de Atividades}
 \label{Rotulo}

\end{figure}

\section{Planejamento da Avaliação}

\chapter{Planejamento das Avaliações}

\begin{figure}[h!]
  \centering
    \includegraphics[keepaspectratio=true, scale=0.5]{figuras/planejamentoavaliacoes.png}
  \caption{Planejamento das avaliações}
\end{figure}

% \begin{figure}[h!]
%   \centering
%     \includegraphics[keepaspectratio=true, scale=0.7]{figuras/listaproblemas.png}
%   \caption{Lista de problemas a ser preenchida nas avaliações}
% \end{figure}
% 
\begin{table*}[!h]
\caption{Lista de problemas a ser preenchida nas avaliações. Fonte: \cite{preece} adaptado}
\label{Rotulo}
  \begin{tabular}{p{0.18\linewidth}p{0.18\linewidth}p{0.30\linewidth}p{0.30\linewidth}}
  \hline
    Nº da Iteração & ID da Questão & Questão & Recomendação\\
 \hline
  \end{tabular}
\end{table*}
  
\pagebreak

\section{Iteração 1}

\begin{figure}[h!]
  \centering
    \includegraphics[keepaspectratio=true, scale=0.7]{figuras/tarefas1.png}
  \caption{Lista de Tarefas para os usuários na Iteração 1}
\end{figure}

\begin{table*}[!h]
\caption{Lista de problemas. Fonte: \cite{preece} adaptado}
\label{Rotulo}
  \begin{tabular}{p{0.18\linewidth}p{0.18\linewidth}p{0.30\linewidth}p{0.30\linewidth}}
  \hline
    Nº da Iteração & ID da Questão & Questão & Recomendação\\
 \hline
  \end{tabular}
\end{table*}
% As avaliações serão feitas permeando todas as fases do processo, assim como sugere o
% Modelo Estrela. Serão escolhidos usuários-chave para uso de protótipos do aplicativo em
% situações pré-determinadas. Na fase de iniciação do projeto, os usuários avaliarão protótipos
% de papel, para levantamento de requisitos. Na execução, com protótipos de alta fidelidade, já
% refinado, esses usuários responderão a questionários referentes a questões de usabilidade, bem
% como serão coletadas informações no momento da avaliação, como reações do usuário
% observadas. Os questionários a serem aplicados, bem como as metas a serem atingidas estão
% descritos neste documento nas seções \ref{questionarios} e \ref{metas}, respectivamente.