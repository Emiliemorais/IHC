\vfill
\chapter[Metas]{Metas de Design}
\label{metas}

Essa seção aborda sobre as metas de design, como metas de usabilidade e metas decorrentes da experiência do usuário 
baseadas na obra de Preece, Rogers e Sharp (\citeyear{preece}).
\section{Metas e Princípios de Usabilidade}

Para essa aplicação as metas de usabilidade a serem alcançadas são: eficácia, eficiência, utilidade e capacidade de 
aprendizado. Pois, o aplicativo deve fazer o que foi proposto para ser feito de forma eficiente, e apresentando as
informações das quais o usuário precisa e sendo intuitivo e fácil de usar. A meta de segurança é importante, todavia o uso 
do aplicativo não inclui tarefas complexas que possam deixar o usuário com medo do que o sistema irá fazer, pois o

aplicativo está mais voltado para funcionalidades de consultas.

No que diz respeito aos princípios de usabilidade para essa aplicação, espera-se que todos os princípios sejam levados em 
consideração. Pois, é muito importante que o usuário possa fazer um bom uso do sistema podendo visualizar o
status do sistema, ter o controle e liberdade, reconhecer e diagnosticar erros e reconhecer objetos do cotidiano que 
se relacionem com atividades na aplicação. Também é importante que o software disponibilize uma ajuda e sua documentação.
\section{Metas Decorrentes da Experiência do Usuário}

Para as metas decorrentes da experiência do usuário espera-se alcançar as seguintes metas: atrativo, satisfação do usuário,
útil e esteticamente apreciável.