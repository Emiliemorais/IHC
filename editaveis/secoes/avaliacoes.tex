\chapter{Avaliações}
Esse capítulo apresenta as avaliações realizadas.

\section{Avaliações do Protótipo de Papel}

\subsection{Relatório Consolidado - Avaliação 1 Protótipo de Papel}

\textbf{Data da avaliação:} 05/04/2015

\textbf{Objetivo:}
O objetivo dessa avaliação consistiu na elicitação de requisitos através de um teste feito em um protótipo de papel.

\textbf{Método:}
O método empregado foi o \textit{Quick and Dirty}\cite{preece}. Foi realizada em um ambiente informal e foi realizada no início do projeto.

\textbf{Usuários e perfil:}
Essa primeira avaliação foi realizada apenas com 1 usuário que se encaixava no público alvo da aplicação proposta. Foi escolhido apenas 1 usuário, pois o objetivo era obter uma rápida avaliação para elicitação de requisitos.

\textbf{Dados coletados:}
Ideias do usuário para uma tela.

\begin{table*}[!h]
\caption{Lista de problemas a ser preenchida nas avaliações. Fonte: \cite{preece} adaptado}
\label{tab:problema}
  \begin{tabular}{p{0.18\linewidth}p{0.18\linewidth}p{0.30\linewidth}p{0.30\linewidth}}
  \hline
    Nº da Iteração & ID da Questão & Questão & Recomendação\\
 \hline
    01 & 01 & Falta de uma tela intermediária entre a página inicial e a página do Matrícula Web & Construir a tela sugerida\\
  \end{tabular}
\end{table*}

\textbf{Planejamento para a próxima versão do protótipo:}
Levando em consideração os problemas encontrados será construída uma nova versão do protótipo. 
\vfill
\pagebreak

\subsection{Relatório Consolidado - Avaliação 2 Protótipo de Papel}

\textbf{Data da avaliação:} 08/05/2015

\textbf{Objetivo:}
O objetivo dessa avaliação consistiu na elicitação de requisitos através de um teste feito em um protótipo de papel.

\textbf{Método:}
O método empregado foi o \textit{Quick and Dirty} \cite{preece}. Foi realizada em um ambiente informal e foi realizada na segunda versão do protótipo construído.

\textbf{Usuários e perfil:}
Essa avaliação foi realizada com 3 usuários que se encaixavam no público alvo da aplicação proposta. Com a continuação do uso do método \textit{Quick and Dirty} os avaliadores optaram por não avaliar com muitos usuários, pois o objetivo ainda era elicitar requisitos. Assim, para não concentrar essa última avaliação do protótipo de papel em apenas um usuário, foi optado pela escolha de 3 usuários.

\textbf{Dados coletados:}
Ideias do usuário para uma tela.

\begin{table*}[!h]
\caption{Lista de problemas a ser preenchida nas avaliações. Fonte: \cite{preece} adaptado}
\label{tab:problema}
  \begin{tabular}{p{0.18\linewidth}p{0.18\linewidth}p{0.30\linewidth}p{0.30\linewidth}}
  \hline
    Nº da Iteração & ID da Questão & Questão & Recomendação\\
 \hline
    02 & 01 & Falta de uma caixa para pesquisa do nome da disciplina & Construir uma caixa de pesquisa\\
  \end{tabular}
\end{table*}

\textbf{Planejamento para a próxima versão do protótipo:}
Levando em consideração os requisitos levantados será construída uma nova versão do protótipo de papel.

\vfill
\pagebreak
\subsection{Relatório Consolidado - Avaliação 3 Protótipo de Papel}

\textbf{Data da avaliação:} 09/06/2015

\textbf{Objetivo:}
O objetivo dessa avaliação consistiu no teste da estabilidade do protótipo de papel.

\textbf{Método:}
O método empregado foi o \textit{Quick and Dirty} \cite{preece}. Foi realizada em um ambiente informal a partir da terceira versão do protótipo construído.

\textbf{Usuários e perfil:}
Essa avaliação foi realizada com 5 usuários que se encaixavam no público alvo da aplicação proposta. 
Foi consentido pelo grupo a escolha de 5 usuários, pois como objetivo era verificar a estabilidade
do protótipo foi levado em consideração o número indicado na literatura.

\textbf{Dados coletados:}

\begin{table*}[!h]
\caption{Lista de problemas. Fonte: \cite{preece} adaptado}
\label{tab:problema}
  \begin{tabular}{p{0.18\linewidth}p{0.18\linewidth}p{0.30\linewidth}p{0.30\linewidth}}
  \hline
    Nº da Iteração & ID da Questão & Questão & Recomendação\\
 \hline
    03 & 01 & Excesso de telas sem real importância & Ver necessidade da tela\\
  \end{tabular}
\end{table*}

\textbf{Planejamento para a próxima versão do protótipo:}
O protótipo será repassado para uma ferramenta.

\vfill
\pagebreak

\section{Avaliações do Protótipo da Ferramenta}

  \subsection{Relatório Consolidado - Avaliação 1 Protótipo Ferramenta}

  \flushleft \textbf{Data da avaliação:} 
  14/06/2015

  \textbf{Objetivo:}
  Avaliar as metas e princípios da Figura \ref{Planejamento}.

  \textbf{Método:}
  O método de avaliação utilizado foi a observação direta, com a aplicação de um questionário.

  \textbf{Usuários e perfil:}
  A avaliação foi feita com 5 (cinco) usuários alvo, ou seja, utilizadores do Matrícula Web.
  
  \textbf{Dados coletados:}

  \begin{itemize}
  \item Lista de problemas da tabela \ref{tab:problema};
  \item Resultado do questionário.
  \end{itemize}


  \textbf{Planejamento para a próxima versão do protótipo:}
  A partir dos resultados obtidos através do questionário, a próxima versão do protótipo deve possuir uma maior visibilidade do status do sistema.

  \vfill
  \pagebreak


  \subsection{Relatório Consolidado - Avaliação 2 Protótipo Ferramenta}

  \flushleft \textbf{Data da avaliação:} 
  19/06/2015

  \textbf{Objetivo:}
  Avaliar as metas e princípios da Figura \ref{Planejamento}.

  \textbf{Método:}
  O método de avaliação utilizado foi a observação direta, com a aplicação de um questionário.

  \textbf{Usuários e perfil:}
  A avaliação foi feita com 5 (cinco) usuários alvo, ou seja, utilizadores do Matrícula Web.

  \textbf{Dados coletados:}

  \begin{itemize}
    \item Resultado do questionário;
    \item Métricas de qualidade.
  \end{itemize}

  \textbf{Planejamento para a próxima versão do protótipo:}
  A partir dos resultados obtidos através do questionário e das métricas de qualidade, conclui-se que o protótipo atende as metas
  e princípios predeterminados. Porém, as notas para o princípio de usabilidade: Reconhecer e diagnosticar erros; indicam que as atividades
  propostas para a avaliação não foram suficientes para a completa análise desse princípio, o que explica sua nota mediana. Contudo, os
  resultados obtidos foram satisfatórios, portanto, não há necessidade de uma nova versão do protótipo.

  \vfill
  \pagebreak
