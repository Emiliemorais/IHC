\chapter[Requisitos]{Requisitos}

\begin{table*}[!h]
\caption{Requisitos Funcionais}
\label{Rotulo}
  \begin{tabular}{p{0.15\linewidth}p{0.45\linewidth}p{0.15\linewidth}p{0.15\linewidth}p{0.15\linewidth}}
  \hline
  Identificador & Descrição & Prioridade & Requisitos Relacionados \\
  \hline

  RF01 & O sistema deve permitir o cadastro de áreas de informação: (Universidade de Brasília) & Alta & Não se aplica\\

  RF02 & O sistema deve permitir o cadastro de usuários, indicando seu nome, e-mail, áreas que deseja curtir, login e senha & Alta & RF01\\

  RF03 & O sistema deve permitir a edição do cadastro do usuário & Alta & RF02\\

  RF04 & O sistema deve permitir a exclusão do cadastro do usuário & Média & RF02\\

  RF05 & O sistema deve permitir que o usuário na área de Universidade de Brasília escolha “Matricula Web” & Alta & RF01 e RF02\\

  RF06 & O sistema deve permitir que o usuário escolha as disciplinas que deseja ver a oferta. & Alta & RF01 e RF02\\

  RF07 & O sistema deve permitir que o usuário programe a hora e a data do recebimento da oferta das disciplinas que escolheu. & Alta & RF06\\

  \hline
  \end{tabular}
% \end{table*}
% 
% \begin{table*}
\caption{Requisitos Não Funcionais}
\label{Rotulo2}
  \begin{tabular}{p{0.15\linewidth}p{0.30\linewidth}p{0.45\linewidth}}
  \hline
  Identificador & Categoria & Descrição\\
  \hline

  RNF01 & Usabilidade & O sistema deve ser fácil de usar, sendo intuitivo ao usuário. (Ver Seção \ref{metas})\\

  RNF02 & Segurança & O sistema deve ser seguro ao usuário, não divulgando informações dos usuários e restringindo o acesso através de login e senha.\\

  RNF03 & Performance & O sistema deve responder as operações realizadas pelo usuário em no máximo 3 segundos.\\

  RNF04 & Portabilidade & O sistema deve funcionar nos sistemas operacionais IOS, Android e Windows Phone.\\


  \hline
  \end{tabular}
\end{table*}
\cleardoublepage

